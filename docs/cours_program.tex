\documentclass[french,]{article}
\usepackage{lmodern}
\usepackage{setspace}
\setstretch{1}
\usepackage{amssymb,amsmath}
\usepackage{ifxetex,ifluatex}
\usepackage{fixltx2e} % provides \textsubscript
\ifnum 0\ifxetex 1\fi\ifluatex 1\fi=0 % if pdftex
  \usepackage[T1]{fontenc}
  \usepackage[utf8]{inputenc}
\else % if luatex or xelatex
  \ifxetex
    \usepackage{mathspec}
  \else
    \usepackage{fontspec}
  \fi
  \defaultfontfeatures{Ligatures=TeX,Scale=MatchLowercase}
  \newcommand{\euro}{€}
\fi
% use upquote if available, for straight quotes in verbatim environments
\IfFileExists{upquote.sty}{\usepackage{upquote}}{}
% use microtype if available
\IfFileExists{microtype.sty}{%
\usepackage{microtype}
\UseMicrotypeSet[protrusion]{basicmath} % disable protrusion for tt fonts
}{}
\usepackage[margin=1in]{geometry}
\usepackage{hyperref}
\PassOptionsToPackage{usenames,dvipsnames}{color} % color is loaded by hyperref
\hypersetup{unicode=true,
            pdftitle={Introduction à la programmation scientifique},
            pdfborder={0 0 0},
            breaklinks=true}
\urlstyle{same}  % don't use monospace font for urls
\ifnum 0\ifxetex 1\fi\ifluatex 1\fi=0 % if pdftex
  \usepackage[shorthands=off,main=french]{babel}
\else
  \usepackage{polyglossia}
  \setmainlanguage[]{french}
\fi
\usepackage{color}
\usepackage{fancyvrb}
\newcommand{\VerbBar}{|}
\newcommand{\VERB}{\Verb[commandchars=\\\{\}]}
\DefineVerbatimEnvironment{Highlighting}{Verbatim}{commandchars=\\\{\}}
% Add ',fontsize=\small' for more characters per line
\usepackage{framed}
\definecolor{shadecolor}{RGB}{248,248,248}
\newenvironment{Shaded}{\begin{snugshade}}{\end{snugshade}}
\newcommand{\KeywordTok}[1]{\textcolor[rgb]{0.13,0.29,0.53}{\textbf{{#1}}}}
\newcommand{\DataTypeTok}[1]{\textcolor[rgb]{0.13,0.29,0.53}{{#1}}}
\newcommand{\DecValTok}[1]{\textcolor[rgb]{0.00,0.00,0.81}{{#1}}}
\newcommand{\BaseNTok}[1]{\textcolor[rgb]{0.00,0.00,0.81}{{#1}}}
\newcommand{\FloatTok}[1]{\textcolor[rgb]{0.00,0.00,0.81}{{#1}}}
\newcommand{\ConstantTok}[1]{\textcolor[rgb]{0.00,0.00,0.00}{{#1}}}
\newcommand{\CharTok}[1]{\textcolor[rgb]{0.31,0.60,0.02}{{#1}}}
\newcommand{\SpecialCharTok}[1]{\textcolor[rgb]{0.00,0.00,0.00}{{#1}}}
\newcommand{\StringTok}[1]{\textcolor[rgb]{0.31,0.60,0.02}{{#1}}}
\newcommand{\VerbatimStringTok}[1]{\textcolor[rgb]{0.31,0.60,0.02}{{#1}}}
\newcommand{\SpecialStringTok}[1]{\textcolor[rgb]{0.31,0.60,0.02}{{#1}}}
\newcommand{\ImportTok}[1]{{#1}}
\newcommand{\CommentTok}[1]{\textcolor[rgb]{0.56,0.35,0.01}{\textit{{#1}}}}
\newcommand{\DocumentationTok}[1]{\textcolor[rgb]{0.56,0.35,0.01}{\textbf{\textit{{#1}}}}}
\newcommand{\AnnotationTok}[1]{\textcolor[rgb]{0.56,0.35,0.01}{\textbf{\textit{{#1}}}}}
\newcommand{\CommentVarTok}[1]{\textcolor[rgb]{0.56,0.35,0.01}{\textbf{\textit{{#1}}}}}
\newcommand{\OtherTok}[1]{\textcolor[rgb]{0.56,0.35,0.01}{{#1}}}
\newcommand{\FunctionTok}[1]{\textcolor[rgb]{0.00,0.00,0.00}{{#1}}}
\newcommand{\VariableTok}[1]{\textcolor[rgb]{0.00,0.00,0.00}{{#1}}}
\newcommand{\ControlFlowTok}[1]{\textcolor[rgb]{0.13,0.29,0.53}{\textbf{{#1}}}}
\newcommand{\OperatorTok}[1]{\textcolor[rgb]{0.81,0.36,0.00}{\textbf{{#1}}}}
\newcommand{\BuiltInTok}[1]{{#1}}
\newcommand{\ExtensionTok}[1]{{#1}}
\newcommand{\PreprocessorTok}[1]{\textcolor[rgb]{0.56,0.35,0.01}{\textit{{#1}}}}
\newcommand{\AttributeTok}[1]{\textcolor[rgb]{0.77,0.63,0.00}{{#1}}}
\newcommand{\RegionMarkerTok}[1]{{#1}}
\newcommand{\InformationTok}[1]{\textcolor[rgb]{0.56,0.35,0.01}{\textbf{\textit{{#1}}}}}
\newcommand{\WarningTok}[1]{\textcolor[rgb]{0.56,0.35,0.01}{\textbf{\textit{{#1}}}}}
\newcommand{\AlertTok}[1]{\textcolor[rgb]{0.94,0.16,0.16}{{#1}}}
\newcommand{\ErrorTok}[1]{\textcolor[rgb]{0.64,0.00,0.00}{\textbf{{#1}}}}
\newcommand{\NormalTok}[1]{{#1}}
\usepackage{graphicx,grffile}
\makeatletter
\def\maxwidth{\ifdim\Gin@nat@width>\linewidth\linewidth\else\Gin@nat@width\fi}
\def\maxheight{\ifdim\Gin@nat@height>\textheight\textheight\else\Gin@nat@height\fi}
\makeatother
% Scale images if necessary, so that they will not overflow the page
% margins by default, and it is still possible to overwrite the defaults
% using explicit options in \includegraphics[width, height, ...]{}
\setkeys{Gin}{width=\maxwidth,height=\maxheight,keepaspectratio}
\setlength{\parindent}{0pt}
\setlength{\parskip}{6pt plus 2pt minus 1pt}
\setlength{\emergencystretch}{3em}  % prevent overfull lines
\providecommand{\tightlist}{%
  \setlength{\itemsep}{0pt}\setlength{\parskip}{0pt}}
\setcounter{secnumdepth}{5}

%%% Use protect on footnotes to avoid problems with footnotes in titles
\let\rmarkdownfootnote\footnote%
\def\footnote{\protect\rmarkdownfootnote}

%%% Change title format to be more compact
\usepackage{titling}

% Create subtitle command for use in maketitle
\newcommand{\subtitle}[1]{
  \posttitle{
    \begin{center}\large#1\end{center}
    }
}

\setlength{\droptitle}{-2em}
  \title{Introduction à la programmation scientifique}
  \pretitle{\vspace{\droptitle}\centering\huge}
  \posttitle{\par}
  \author{}
  \preauthor{}\postauthor{}
  \predate{\centering\large\emph}
  \postdate{\par}
  \date{17 avril 2016}


\usepackage{times}
\usepackage{fancyhdr}
\pagestyle{fancy}
\fancyhead[CO,CE]{}
\fancyfoot[CO,CE]{Introduction à la programmation scientifique}
\fancyfoot[LE,RO]{\thepage}

% Redefines (sub)paragraphs to behave more like sections
\ifx\paragraph\undefined\else
\let\oldparagraph\paragraph
\renewcommand{\paragraph}[1]{\oldparagraph{#1}\mbox{}}
\fi
\ifx\subparagraph\undefined\else
\let\oldsubparagraph\subparagraph
\renewcommand{\subparagraph}[1]{\oldsubparagraph{#1}\mbox{}}
\fi

\begin{document}
\maketitle

{
\setcounter{tocdepth}{3}
\tableofcontents
}
\newpage

\section{Introduction à la programmation
scientifique}\label{introduction-a-la-programmation-scientifique}

Programmer c'est trop cool !

\newpage

\section{L'interface R}\label{linterface-r}

Ici les ressources ?

\subsection{Historique}\label{historique}

\subsection{Types d'objets}\label{types-dobjets}

\subsection{Lire des fichiers}\label{lire-des-fichiers}

\subsection{Écrire des fichiers}\label{ecrire-des-fichiers}

\subsection{Faire des scripts}\label{faire-des-scripts}

\begin{Shaded}
\begin{Highlighting}[]
\NormalTok{vec1 <-}\StringTok{ }\KeywordTok{c}\NormalTok{(}\DecValTok{2}\NormalTok{,}\DecValTok{3}\NormalTok{,}\DecValTok{8}\NormalTok{)}
\NormalTok{vec2 <-}\StringTok{ }\DecValTok{1}\NormalTok{:}\DecValTok{10}
\NormalTok{vec3 <-}\StringTok{ }\KeywordTok{rep}\NormalTok{(}\KeywordTok{c}\NormalTok{(}\DecValTok{1}\NormalTok{,}\DecValTok{3}\NormalTok{),}\DecValTok{4}\NormalTok{)}
\NormalTok{vec4 <-}\StringTok{ }\KeywordTok{seq}\NormalTok{(}\DecValTok{1}\NormalTok{,}\DecValTok{10}\NormalTok{,}\FloatTok{0.5}\NormalTok{)}
\KeywordTok{print}\NormalTok{(vec1)}
\end{Highlighting}
\end{Shaded}

\begin{verbatim}
## [1] 2 3 8
\end{verbatim}

deuxième bout de code

\begin{Shaded}
\begin{Highlighting}[]
\NormalTok{vec5 <-}\StringTok{ }\KeywordTok{rep}\NormalTok{(LETTERS[}\DecValTok{1}\NormalTok{:}\DecValTok{4}\NormalTok{], }\DataTypeTok{each=}\DecValTok{2}\NormalTok{)}
\KeywordTok{print}\NormalTok{(vec5)}
\end{Highlighting}
\end{Shaded}

\begin{verbatim}
## [1] "A" "A" "B" "B" "C" "C" "D" "D"
\end{verbatim}

\subsection{Charger des librairies}\label{charger-des-librairies}

\newpage

\section{Manipulation des données}\label{manipulation-des-donnees}

\subsection{indexation des objets}\label{indexation-des-objets}

\begin{Shaded}
\begin{Highlighting}[]
\NormalTok{df1 <-}\StringTok{ }\KeywordTok{data.frame}\NormalTok{(}\DataTypeTok{var1=}\KeywordTok{runif}\NormalTok{(}\DecValTok{10}\NormalTok{), }\DataTypeTok{var2=}\KeywordTok{rnorm}\NormalTok{(}\DecValTok{10}\NormalTok{))}
\KeywordTok{print}\NormalTok{(df1)}
\end{Highlighting}
\end{Shaded}

\begin{verbatim}
##          var1         var2
## 1  0.09876419  0.268601796
## 2  0.80504843  0.577963613
## 3  0.60757396 -0.533107748
## 4  0.38283221 -1.090589273
## 5  0.99780371  0.907842364
## 6  0.99242618 -0.509955984
## 7  0.38290775  0.105750555
## 8  0.23273107 -0.006120552
## 9  0.59650867 -1.717606296
## 10 0.34567095  0.448090674
\end{verbatim}

\subsection{split}\label{split}

\subsection{merge}\label{merge}

\subsection{tableau croisé}\label{tableau-croise}

\newpage

\section{Outils mathématiques de
base}\label{outils-mathematiques-de-base}

\subsection{opérations de base}\label{operations-de-base}

\subsection{sommes, produits}\label{sommes-produits}

\begin{Shaded}
\begin{Highlighting}[]
\CommentTok{# une addition avec R.}
\DecValTok{2+3}
\end{Highlighting}
\end{Shaded}

\begin{verbatim}
## [1] 5
\end{verbatim}

\begin{Shaded}
\begin{Highlighting}[]
\CommentTok{# une somme sur un vecteur}
\KeywordTok{sum}\NormalTok{(}\KeywordTok{c}\NormalTok{(}\DecValTok{2}\NormalTok{,}\DecValTok{5}\NormalTok{,}\DecValTok{1}\NormalTok{,}\DecValTok{9}\NormalTok{))}
\end{Highlighting}
\end{Shaded}

\begin{verbatim}
## [1] 17
\end{verbatim}

\subsection{produit matriciel}\label{produit-matriciel}

\subsection{analyse de vecteurs
propres}\label{analyse-de-vecteurs-propres}

\newpage

\section{Fonctions}\label{fonctions}

\subsection{environnemnts}\label{environnemnts}

\newpage

\section{Récursions}\label{recursions}

\subsection{For loop}\label{for-loop}

\subsection{While}\label{while}

\subsection{apply (et autres)}\label{apply-et-autres}

\newpage

\section{Algorithmes}\label{algorithmes}

\subsection{Énoncés conditionnels}\label{enonces-conditionnels}

\subsection{Planifier une séquence
d'opérations}\label{planifier-une-sequence-doperations}

\newpage

\section{Processus stochastiques}\label{processus-stochastiques}

\newpage

\section{Optimisation des scripts}\label{optimisation-des-scripts}

\end{document}
